\section{Modulo: Modelo de Autos}

\subsection{UC05. Crear un Modelo de Auto} \label{CrearModeloAuto}
\textbf{Actores}: Administrador

\textbf{Objetivo}: EU-Rent decide ofrecer a sus Clientes un nuevo modelo de Auto.

\textbf{Evento Disparador}: Administrador solicita la p�gina \textit{[Crear Modelo de Auto]}.

\textbf{Tipo}: Usuario\\

\textbf{Escenario Principal}

\begin{enumerate}
\item El Sistema muestra la p�gina \textit{[Crear Modelo de Auto]}.
\item Administrador captura informaci�n en la forma \textit{[Crear Modelo de Auto]}.
\item Administrador elige el comando \textit{[Guardar]}.
\item El Sistema valida que los datos de la forma \textit{[Crear Modelo de Auto]} estan completos.
	\begin{enumerate}
		\item Excepci�n: Datos incompletos.
	\end{enumerate}
\item El Sistema crea un nuevo registro en la entidad \textit{[ModeloAuto]}.
\item Fin del Caso de Uso.
\end{enumerate}
\subsection{UC06. Buscar un Modelo de Auto} \label{BuscarModeloAuto}
\textbf{Actores}: Administrador

\textbf{Objetivo}: EU-Rent decide buscar un Modelo de Auto para editarlo o eliminarlo.

\textbf{Evento Disparador}: Administrador solicita la p�gina \textit{[Buscar Modelo de Auto]}.

\textbf{Tipo}: Usuario\\

\textbf{Escenario Principal}

\begin{enumerate}
\item El Sistema muestra la p�gina \textit{[Buscar Modelo de Auto]}.
\item Administrador captura informaci�n en la forma \textit{[Criterios de B�squeda]}.
\item Administrador elige el comando \textit{[Buscar]}.
\item El Sistema valida que los datos de la forma \textit{[Criterios de B�squeda]} estan completos.
	\begin{enumerate}
		\item Excepci�n: Datos incompletos.
	\end{enumerate}
\item El Sistema obtiene informaci�n y muestra la lista \textit{[Resultados de B�squeda]}.
\item Fin del Caso de Uso.
\end{enumerate}
\subsection{UC07. Editar un Modelo de Auto} \label{EditarModeloAuto}
\textbf{Actores}: Administrador

\textbf{Objetivo}: EU-Rent decide modificar los datos de un Modelo de Auto.

\textbf{Evento Disparador}: Administrador solicita la p�gina \textit{[Editar Modelo de Auto]}.

\textbf{Tipo}: Usuario\\

\textbf{Escenario Principal}

\begin{enumerate}
\item El Sistema invoca al Caso de Uso \textit{[UC06. Buscar un Modelo de Auto]}.
\item Administrador selecciona el comando \textit{[Editar]}.
\item El Sistema muestra la p�gina \textit{[Editar Modelo de Auto]}.
\item Administrador captura informaci�n en la forma \textit{[Editar Modelo de Auto]}.
\item Administrador elige el comando \textit{[Guardar]}.
\item El Sistema valida que los datos de la forma \textit{[Editar Modelo de Auto]} estan completos.
	\begin{enumerate}
		\item Excepci�n: Datos incompletos.
	\end{enumerate}
\item El Sistema actuliza la informaci�n de la entidad \textit{[ModeloAuto]}.
\item Fin del Caso de Uso.
\end{enumerate}
\subsection{UC08. Eliminar Modelo de Auto} \label{EliminarModeloAuto}
\textbf{Actores}: Administrador

\textbf{Objetivo}: EU Rent decide eliminar un Modelo de Auto existente.

\textbf{Evento Disparador}: Administrador solicita la p�gina \textit{[Eliminar Modelo de Auto]}.

\textbf{Tipo}: Usuario\\

\textbf{Escenario Principal}

\begin{enumerate}
\item El Sistema invoca al Caso de Uso \textit{[UC06. Buscar un Modelo de Auto]}.
\item Administrador selecciona el comando \textit{[Eliminar]}.
\item El Sistema muestra la p�gina \textit{[Eliminar Modelo de Auto]}.
\item Administrador elige el comando \textit{[Eliminar]}.
\item El Sistema elimina el registro de la entidad \textit{[ModeloAuto]}.
\item Fin del Caso de Uso.
\end{enumerate}
